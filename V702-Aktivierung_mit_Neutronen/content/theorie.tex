\newpage
\section*{Zielsetzung}
Bestimmung von Halbwertszeiten verschiedener Isotopen.
\section{Theorie}
Ist das Verhätnis von Neutroenen- und Protonenzahl nictn innerhalb einer engen Grenze, so kann sich
ein Atomkern mit einer bestimmenten Wahrscheinlichkeit in einen stabilen oder in einen anderen
instabilen Kern wandeln. Eine typische emprirische Darstellung für die Wahrscheinlichkeit einer
solchen Umwandlung drückt die Halbwertszeit $T$ aus. Sie beschreibt den Zeitraum in dem eine Anzahl von stabilen Kernen
zur Hälfe Zerfallen sind.\\
Ursache für einen solchen Zerfall kann die wechselwirkung mit einem Neutron sein.
\subsection{Neutronen als Ursprung von Kernreaktionen}
Wird ein Atomkern mit einem Neutrone beschossen, so führt die Absorbtion dessen zu einer Veränderung
des Ursprungkerns A zu einem Zwischenkern (bzw. Compoundkern) A*.
